{\indent Scientific publications present research as a straight-line progression from questions to conclusions. In contrast, a PhD program feels like a winding journey, full of unexpected lessons learned along the way. Throughout this interesting time, I have benefited from the guidance of several amazing scientists. 
	
My first research experience was with Prof. Sean Scully, while I was still an undergraduate. During my senior year, Francis-Yan Cyr-Racine, Leonidas Moustakas, and Chuck Keeton brought me to do research with them at JPL, during which time I was introduced to the subject matter described in this dissertation. The opportunity to collaborate with more experienced scientists at these early stages was invaluable. I am grateful for the opportunity they gave me, and their continued support today. In particular, since day one I have collaborated closely with Anna Nierenberg. Learning from her over the last few years has made me a better scientist. 

I was truly fortunate to have the chance to work closely with Simon Birrer, who started mentoring me at UCLA almost immediately after finishing his own PhD. I imagine it was a challenge to transition so quickly from student to supervisor, but it seemed natural for him. If I am ever put in a mentoring role myself and am unsure how to proceed, I'll just think to myself: What would Simon do in this situation? 

It is difficult to convey just how much I admire my PhD advisor, Tommaso Treu. The astronomy community at large regards him as a creative and talented scientist. To me, he is also a role model, and a leader. I could not have asked for a better advisor.  

Finally, I am thankful for the support from my amazing parents. It must have been interesting listening to me try to choose a career path: from a pilot, to joining the foreign service, and then finally setting down with astrophysics. Through it all, you challenged me and believed in me. As I get older, I realize more and more how lucky I am to have you two as my parents - I could not have done it without you! 
}