
%
% introduction.tex
% Copyright (C) 1995 by John Heidemann, <johnh@isi.edu>.
% $Id: demo2int.tex,v 1.1 1996/01/12 18:13:58 johnh Exp $
%

\chapter{Introduction}

\indent Approximately 9 billion years ago, the quasar WFI2033-4723 ejected four photons in slightly different directions. Left to fly freely, a distance of roughly $100,000$ light years would separate them at the present time. However, today we find the photons collected at the same position, on the primary mirror of the Hubble Space Telescope. 

The gravitational field of a massive galaxy in the foreground, precisely aligned between us and WFI2033-4723, intervened to deflect the light emitted from the quasar; this phenomenon is referred to as gravitational lensing. The warping of space around the foreground galaxy connects Hubble with the background source in four different ways, so, we see the quasar WFI2033-4723 as it if were situated at four different positions on the sky. The background source, the intervening mass, and the resulting image of WFI2033-4723, displayed in the bottom-central panel of Figure \ref{fig:lens2033}, constitute a quadruple-image \textit{strong} gravitational lens system, or a \textit{quad}. The adjective `strong' indicates the appearance of multiple images of the background source, and occasionally luminous arcs that encircle main deflector. 

Strong gravitational lensing provides a means of connecting observable quantities, such as the arrival times, positions, and brightness of lensed images, with all of the gravitating mass along the line of sight from the observer to the source. Crucially, lensing does not discriminate between dark matter and baryonic matter; as far as lensing is concerned, all matter is created equal. The direct gravitational coupling of lensing observables to mass enables a direct probe of otherwise undetectable dark matter structure across distances than span most of the observable Universe.    

If the reigning cosmological theory of Cold Dark Matter (CDM), the bedrock of modern cosmology, is correct, low-mass and completely invisible dark matter halos should litter the cosmos. If present, these otherwise-invisible dark matter structures should leave measurable imprints on strong lensing observables, particularly on the image magnifications in quads. A positive detection of low-mass halos through lensing would confirm a central prediction of CDM. A non-detection of low-mass halos would demand an explanation for their absence, potentially overthrowing entire classes of particle physics models that predict a plethora of dark matter structure in the Universe. 

\begin{figure*}
	\centering
	\includegraphics[clip,trim=2.5cm 8cm 2.5cm
	8.5cm,width=.9\textwidth,keepaspectratio]{./figures_introduction/lenses.pdf}
	\caption{\label{fig:lens2033} Six quadruple-image strong gravitational lens systems imaged by the Hubble Space Telescope \cite{Nierenberg++19}. The main lensing galaxy is visible as the faint object encircled by four images of a background quasar. Image credit: NASA, ESA, A. Nierenberg (JPL) and T. Treu (UCLA)}
\end{figure*}	
In this dissertation, I describe research that constrains particle theories of dark matter with strong gravitational lensing. The following sections of this introduction set the stage for Chapters 2-6, which describe the development and implementation of a technique to directly constrain any dark matter model with the image magnifications from a sample of quadruple-image strong lenses. Section 1.1 begins with a review of how the particle nature of dark matter drives structure formation in the Universe, and what aspects of structure formation lensing can constrain. Next, I review the basic theory connecting dark matter structure to lensing observables. 

\section{Structure formation and dark matter physics}
\indent Particle theories of dark matter predict the eventual collapse of an initially diffuse field of dark matter particles into gravitationally-bound structures, or halos, through a mechanism called `violent relaxation' \cite{LyndenBell67}. The halo mass function, or the number of collapsed structures per unit mass, encodes information about when the first dark matter halos collapsed in the early Universe. Similarly, the density profiles of individual halos as a function of mass, the mass-concentration relation, depends on the hierarchical assembly history of dark matter halos through cosmic time, and the shape of the primordial matter power spectrum that seeded their growth.

Both the halo mass function and mass-concentration relation depend directly on the particle nature of dark matter. As a concrete example, consider two competing classes of dark matter models: Cold, and Warm Dark Matter (CDM and WDM, respectively). The free-streaming length $\lambda_{FS}$ distinguishes CDM from WDM. Free-streaming refers to the diffusion of dark matter particles out of potential wells in the early Universe, before structure begins growing in earnest around the time of matter-radiation equality $t_{\rm{EQ}}$. One definition of $\lambda_{FS}$ specifies this quantity as the comoving distance a particle could have traveled before $t_{\rm{EQ}}$ \cite{Schneider++12} 
\begin{figure}
	\centering
	\includegraphics[clip,trim=0cm 0cm 0cm
	0cm,width=.48\textwidth,keepaspectratio]{./figures_introduction/CDMscreenshot_edited.pdf}
	\includegraphics[clip,trim=0cm 0cm 0cm
	0cm,width=.48\textwidth,keepaspectratio]{./figures_introduction/WDMrealization_nobar.pdf}
	\caption{\label{fig:wdmrealization} {\bf{Left:}} A realization of CDM substructure, with a scale-free subhalo mass function. {\bf{Right:}} A realization of WDM substructure corresponding to a $3.3 \rm{keV}$ thermal relic dark matter particle. Free-streaming in WDM produces a turnover in the halo mass function, erasing structure on small scales.}
\end{figure}

\begin{equation}
\label{eqn:freestreaming}
\lambda_{FS} \approx \int_{0}^{t_{\rm{NR}}} \frac{c dt}{a\left(t\right)} +  \int_{t_{\rm{NR}}}^{t_{\rm{EQ}}} \frac{v\left(t\right) dt}{a\left(t\right)} \approx r_{H}\left(t_{\rm{NR}}\right) \left(1 +\frac{1}{2} \log \frac{t_{\rm{EQ}}}{t_{\rm{NR}}} \right)
\end{equation}
where the particle has speed $c$ before becoming non-relativistic at time $t_{NR}$, $r_H \left(t_{\rm{NR}}\right)$ is the comoving horizon size at $t_{\rm{NR}}$, $a\left(t\right)$ is the cosmological scale factor, and $v\left(t\right)$ represents the average velocity distribution of the dark matter particles\footnote{This expression assumes that the particles become non-relativistic before $t_{\rm{EQ}}$, and uses the fact that $a\left(t\right)\propto t^{\frac{1}{2}}$ before $t_{\rm{EQ}}$.}. Free-streaming wipes out small-scale density fluctuations, transforming a density field initialized with a scale-free power spectrum $P\left(k\right) \propto k^{n}$ into a density field with a truncated power spectrum around $k \sim k_{\rm{FS}} = \frac{2 \pi}{\lambda_{FS}}$. 

The effects of free-streaming manifest in structure formation in two ways: First, erasing small-scale power at early times eliminates the small-scale density perturbations that would collapse into low-mass halos. The result is a turnover in the halo mass function \cite{AvilaReese++01,Schneider++12,Lovell++14}, and a dearth of structure below a halo mass scale that is proportional to the $k_{\rm{FS}}^{-3}$. Second, eliminating the smallest halos delays the onset of structure formation, because low-mass halos collapse first in hierarchical structure formation scenarios. As the central density profile of dark matter halos reflects the density of the Universe at the time of collapse, delaying structure formation also suppresses the concentration-mass relation of halos near the free-streaming scale in WDM cosmologies. In fact, the suppression of halo concentrations extends over an order of magnitude in mass above the free-streaming scale \cite{Navarro++96,Bose++16}. Examples of CDM and WDM substructure are shown in the left and right panels of Figure \ref{fig:wdmrealization}, respectively. 

The dependence of $\lambda_{FS}$ on features such as $t_{NR}$ and $v\left(t\right)$ links the free-streaming length of the dark matter to the formation mechanism and velocity distribution of the dark matter particle(s). As the halo mass function and concentration-mass relation depend on the free-streaming length, it follows that constraining the halo mass function and halo density profiles can be cast as a constraint on fundamental dark matter physics determining $t_{NR}$ and $v\left(t\right)$. 

Notice that none of the previous discussion depends on a particular choice of dark matter model; properties such as the free-streaming length can be computed for practically any model in the literature. The power of structure formation arguments lies in their broad scope and applicability to several different dark matter models at once. For example we can rule out neutrinos, thermal relics with mass $< 2 \rm{keV}$, and sterile neutrinos produced via Higgs decay with a mass of $7 \rm{keV}$ \cite{Viel13,AbazaijanKusenko19} making up 100$\%$ of the dark matter, as the free-streaming lengths corresponding to each these models precludes the formation of galaxies such as the Milky Way. 

In order to constrain dark matter models through structure formation arguments, one requires a method to detect, and measure the mass of, dark matter halos. One approach uses the fact that, according to CDM, galaxies live inside dark matter halos. One could therefore use luminous structures, such as dwarf galaxies,  to trace for the underlying dark matter. However, this approach becomes increasingly difficult below $10^9$ solar masses, as not every dark matter halo with this mass hosts a visible galaxy. It is therefore difficult to determine, using luminous matter to trace the dark matter, whether low-mass halos simply do not contain galaxies, or whether the low-mass halos do not exist. The uncertainties stemming from sub-galactic astrophysics can sometimes be larger that the differences between the dark matter models of interest \cite{Nierenberg++16}. While the latest advances in this field make considerable progress towards appropriately dealing with these complications \cite{Nadler++19}, they are, at present, limited to the Milky Way. A second technique to probe small-scale structure in the Universe relies on the flux power spectrum of the Lyman-$\alpha$ forest at $z \sim 5$ \cite{Viel13,Irsic++17}. The promise of this method must be weighed against the systematic uncertainties associated with the thermodynamics of the Lyman-$\alpha$ forest at high redshift \cite{Garzilli++19}. 

Strong gravitational lensing by galaxies offers an alternative, more direct probe of dark matter structure on scales below $10^8$ solar masses. Lensing couples only to gravity, and therefore circumvents the challenges associated with using baryonic matter to trace the underlying dark matter. In the next section, I review the formalism connecting lensing observables to populations of dark matter halos along the entire line of sight, from the observer to the source. 

\section{Strong lensing signatures of dark matter halos}
\indent General relativity relates the deflection angle of a light ray to the mass distribution of a massive structure, dark or luminous. When the distances scales between the observer, lens, and source are much greater than the physical extent of the lensing mass distribution, the effect of a massive deflector can be approximated as a single sharp deflection in the plane of the lens, the `thin lens' approximation. Defining $\Sigma$ as the projection of a deflector's three dimensional density profile onto the plane of the lens, the deflection angle is given by \cite{BlandfordNarayan86,Schnedier1997}
\begin{equation}
\label{eqn:defangle}
\vec{\alpha}\left(D_{\rm{d}} \vec{\theta}\right) = \frac{4G D_{\rm{d}}}{c^2} \int \frac{\left(\vec{\theta} - \vec{\theta}^{\prime}\right) \Sigma\left(D_{\rm{d}}\vec{\theta}^{\prime}\right)}{|\vec{\theta} - \vec{\theta}^{\prime}|^2} d^2 \vec{\theta^{\prime}}. 
\end{equation}
For multiple deflectors in a single lens plane, the cumulative effect of multiple deflectors is a linear superposition of their individual deflection angles $\vec{\alpha}$. 

A strong lens system will include both subhalos associated with the host dark matter halo of the lensing galaxy, and field matter halos distributed along the entire line of sight. Incorporating line of sight halos requires multi-plane ray tracing equation, which maps an angular coordinate on the sky $\vec{\theta_1}$ to an angular coordinate on the source plane $\vec{\theta_s}$. The ray tracing equation is given by \cite{BlandfordNarayan86}
\begin{equation}
\label{eqn:raytracingintro}
\vec{\theta_s} = \vec{\theta_1} - \frac{1}{D_{\rm{s}}} \sum_{i=1}^{s-1} D_{\rm{is}}{\vec{\alpha_{\rm{i}}}} \left(D_{\rm{i}} \vec{\theta_{\rm{i}}}\right),
\end{equation} 
where the net deflection angle from all halos at the $i$th lens plane can be computed with Equation \ref{eqn:defangle}, and $D_{\rm{ij}}$ is the angular diameter distance from the $i$th lens plane to the $j$th, and subscript $s$ identifies the source plane. Equation \ref{eqn:raytracingintro} is a recursive equation for the net deflection angles at various lens planes, coupling deflections produced by halos at different distances. Equation \ref{eqn:raytracingintro} describes a physical process similar to viewing an image through multiple magnifying glasses in series. 

Equation \ref{eqn:raytracingintro} maps a single coordinate on the sky to a single coordinate on the source plane, determining where images appear to the observer. As gravitational lensing conserves surface brightness \cite{MisnerThorneWheeler}, the (de)magnification of a lensed image is proportional to the ratio of the areas in the image and source planes. The ratio of areas is given by the inverse determinant of the jacobian $\left(\det \frac{\partial \vec{\theta_s}}{\partial \vec{\theta_1}}\right)^{-1}$. While the full expression for the lensing jacobian in the general multi-plane framework (see \cite{BlandfordNarayan86}) is long and not particularly illuminating, the key point is that the magnification of an image depends non-linearly on derivatives of the lensing deflection angle. Image magnifications are therefore highly localized probes of the mass distribution along the line of sight to strong lenses. While the exact level of perturbation to an image magnification depends on the size of the background source, the mass of the halo, and the position of the image relative to the critical curve, a dark matter halo near a lensed image can induce measurable perturbations on image magnifications even if it is orders of magnitude less massive that the deflector producing the multiple images. 

The idea that dark matter halos frequently perturb image magnifications was first put forward in 1997 \cite{MaoSchneider98}. Since that time, authors have attributed lensing `flux anomalies', or the consistent failure of smoothly-parameterized mass distributions to reproduce the magnifications ratios observed in quad lens systems\footnote{Since the intrinsic brightness of the source is unknown, the observable quantity is the magnification ratio, rather than the magnification itself.}, to the presence of substructure in the lens system \cite{Metcalf++02,D+K02,Xu++12,Xu++15}. Early studies of strong lensing flux anomalies (e.g. \cite{D+K02}) relied on lensed radio emission from the background quasar. This technique has two drawbacks that are remedied by the advent of nuclear narrow-line emission from the background quasar as probe of substructure, a method first proposed by \cite{MoustakasMetcalf02}, and subsequently implemented by \cite{Sugai++07,Nierenberg++14,Nierenberg++17,Nierenberg++19}. The use of lensed narrow-line emission has two advantages: First, the nuclear narrow-line region is spatially extended by $\sim 50$pc \cite{MullerSanchez++11}, which avoids contamination from stellar microlensing that can affect magnifications measured in radio wavelengths. Second, narrow-line emission is measurable in virtually every quasar, expanding the sample size of available lens systems. Recently, the sample size of strong lens systems with measured narrow-line flux ratios increased by nearly a factor of four \cite{Nierenberg++19}. 

The image magnifications from these systems provide a reliable dataset with which to test key predictions of CDM, including the shape of the halo mass function and the mass-concentration relation of CDM halos. During my PhD, I developed methods to use these data to place one of the tightest constraints on the free-streaming length of dark matter to date, independent of and more stringent than those obtained from the Lyman-$\alpha$ forest \cite{Viel13}. Using the tools I developed, I also placed the first observational constraint on the mass-concentration relation of CDM halos on sub-galactic scales across cosmological distance. This dissertations presents the development of these methods: Chapter 1 describes a study that quantifies the intrinsic uncertainty associated with smoothly-parameterized lensing mass profiles, irrespective of the dark matter substructure content of the lens system. Chapters 2 and 3 present papers that describe the development and testing of the analysis framework I developed. Finally, in Chapters 5 and 6 I summarize the results obtained from using this framework to constraining the free-streaming length of dark matter, and the concentration-mass relation of CDM halos, respectively. 

% \section{Related Work}
%	\label{sec:intro_related_work}

% ...


% LocalWords:  Posix Novell's Netware Kernighan Madnick Alsop NeFS PostScript
% LocalWords:  Rosenthal SunSoft NEEDSWORK Wong
