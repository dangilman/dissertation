Dark matter makes up most of the mass in the Universe, and yet its particle nature remains unknown. Structure formation arguments provide a promising avenue to address this confounding mystery, as the mass and formation mechanism of the dark matter manifests in the abundance and density profiles of dark matter halos. Measurements of the halo mass function and the mass-concentration relation can therefore be cast as direct constraints on the particle nature of dark matter itself.

Strong gravitational lensing by galaxies offers a unique probe of dark matter structure across cosmological distance, circumventing the use of luminous matter to trace the underlying dark matter. Observables from strong lens systems, particularly the image magnifications in quadruply-imaged quasars, probe the halo mass function directly on sub-galactic scales, below $10^8$ solar masses. In this low-mass regime, where halos become devoid of stars and gas, various dark matter models make unique predictions that lensing can constrain. 

In this dissertation, I present the development and implementation of a forward modeling framework that constrains any model based on dark matter theory, provided the model predicts the form of the halo mass function, and the density profile of individual halos. Using the framework I developed, my thesis presents an unprecedented constraint on the free-streaming length of dark matter that corresponds to a lower limit of $5.2 \rm{keV}$ on the mass of a thermal relic dark matter particle. In addition, I present the first constraint on the mass-concentration relation of Cold Dark Matter halos on sub-galactic scales across cosmological distance. The flexibility of the framework I developed broadens the scope of strong-lensing analyses to any structure formation model based on dark matter theory, underscoring the power of strong gravitational lensing as a probe of fundamental physics. 