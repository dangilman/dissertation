Dark matter makes up most of the mass in the Universe, and yet its particle properties remain unknown. As dark matter drives structure formation in the Universe, clues regarding the particle nature of dark matter are imprinted in the abundance and density profiles of dark matter halos. Measurements of the halo mass function and the concentrations of individual dark matter halos, particularly on sub-galactic scales, can therefore be cast as direct constraints on the particle nature of dark matter.

Strong gravitational lensing by galaxies offers a unique probe of dark matter structure in the Universe across cosmological distance. By coupling directly to gravity, lensing probes structure without relying on luminous matter as a tracer of the underlying dark matter. The direct gravitatioanl coupling also extends the reach of lensing probes below $10^8$ solar masses, where halos are mostly devoid of stars and gas, and where various dark matter theories make divergent predictions for the halo mass function and the mass-concentration relation of halos. 

In this dissertation, I present the development of a forward modeling framework to constrain any model based on dark matter theory, provided the model makes predictions for the form of the halo mass function and the density profile of individual dark matter halos. The formalism I present handles the marginalization over nuisance parameters, including the mass profile of the main deflector and finite source effects, in a fully Bayesian framework, and accounts for both subhalos associated with the main deflector and field halos along the line of sight. 

Using the framework I developed, my thesis presents an unprecedented constraint on the free-streaming length of dark matter that corresponds to a lower limit of $5.2 \rm{keV}$ on the mass of a thermal relic dark matter particle. In addition, I present the first measurement of the mass-concentration relation of Cold Dark Matter halos on sub-galactic scales across cosmological distance. 