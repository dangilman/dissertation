\chapter{Conclusions}

In my thesis, I developed techniques that synthesize data from quadruply-imaged quasars to place constraints on any dark matter model by constraining the shape of the halo mass function, and the density profiles of individual dark matter halos. The multi-year effort of developing the necessary analysis tools culminated in two main results. First, we placed a an upper bound on the free-streaming length of $0.044 \ \rm{Mpc} \ h^{-1}$, assuming a mass function based a thermal relic dark matter particle. The corresponding constraint on the particle mass sets a lower bound of $5.2 \rm{keV}$, one of the strongest constraints on WDM to date. Second, we placed the first observational constraint on the mass-concentration relation of cold dark matter halos on sub-galactic scales across cosmological distance, finding that models in the literature are all consistent with the observations. Neither of these results would have been possible without the pioneering work to measure narrow-line flux ratios presented in \cite{Nierenberg++19}. 

The constraints on the free-streaming length and the mass-concentration arrived quickly on each other's heels, published only a few months apart. Once the analysis framework, which took years to develop, was in place, it was rapidly deployed to test two quite different parameterizations of the mass function and halo density profiles. This illustrates both the power of strong lensing as a probe of fundamental physics, and the flexibility of the tools I developed to efficiently exploit this power. 

The results I have presented in this dissertation suggest new avenues to pursue. The flexibility of my method accommodates any parameterization of halo density profiles, including the cored profiles characteristic of self-interacting and ultra-light `fuzzy' dark matter, as well as the intrinsically point-like mass profiles characteristic of primordial black holes. Strong lensing could conceivably constrain each of these models. In coming years, the constraining power of strong lensing analyses over fundamental dark matter physics will increase in tandem with the continuously-growing sample size of strongly-lensed quasars with reliable, high-quality data. This enhanced constraining power may deliver constraints on the amplitude of subhalo mass function tight enough to test the tidal stripping processes that determined the observed number of dwarf galaxies. Finally, although strong lensing is often presented as direct competitor with `rival' techniques such as the Lyman-$\alpha$ forest and stellar streams, the strongest constraints will result from analyses that combine each of these independent probes to place joint constraints on the structure formation processes that hold clues regarding the nature of dark matter . 
