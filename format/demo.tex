
\documentclass [PhD] {uclathes}

% \def\aap{AA}
\def\prl{Phys. Rev. Lett.}
\def\aapr{AA Rev}
\def\apjl{ApJL}
\def\na{NewA}
\def\apss{APSS}
\def\mnras{MNRAS}
\def\apj{ApJ}
\def\apjs{ApJS}
\def\aj{AJ}
\def\pasp{PASP}
\def\pasj{PASJ}
\def\nat{Nat}
\def\memsai{MmSAI}
\def\prd{PhysRevD}
\def\jcap{JCAP}
\def\physrep{Phys. Rept.}
\def\araa{Annual Review of Astronomy and Astrophysics}
\newcommand{\newblock}{}                         % personal LaTeX macros

%%%%%%%%%%%%%%%%%%%%%%%%%%%%%%%%%%%%%%%%%%%%%%%%%%%%%%%%%%%%%%%%%%%%%%
%
% Usually things live in separate flies.
%
% \input {prelim}                           % preliminary page info

%%%%%%%%%%%%%%%%%%%%%%%%%%%%%%%%%%%%%%%%%%%%%%%%%%%%%%%%%%%%%%%%%%%%%%%%
%                                                                      %
%                          PRELIMINARY PAGES                           %
%                                                                      %
%%%%%%%%%%%%%%%%%%%%%%%%%%%%%%%%%%%%%%%%%%%%%%%%%%%%%%%%%%%%%%%%%%%%%%%%

\title          {Investigating the Nature of Dark \\
                Matter with Strong Gravitational Lensing}
\author         {Daniel Alejandro Gilman}
\department     {Astrophysics}
% Note:  degreeyear should be optional, but as of  5-Feb-96
% it seems required or you get a year of ``2''.   -johnh
\degreeyear     {2020}

%%%%%%%%%%%%%%%%%%%%%%%%%%%%%%%%%%%%%%%%%%%%%%%%%%%%%%%%%%%%%%%%%%%%%%%%

\chair          {Tommaso L.\ Treu}
\member         {Alex Kusenko}
\member         {Smadar Naoz}
\member         {Steven R.\ Furlanetto}

%%%%%%%%%%%%%%%%%%%%%%%%%%%%%%%%%%%%%%%%%%%%%%%%%%%%%%%%%%%%%%%%%%%%%%%%

\dedication     {\textsl{To my parents, Ivelisse and David, for their unwavering love and support.}}

%%%%%%%%%%%%%%%%%%%%%%%%%%%%%%%%%%%%%%%%%%%%%%%%%%%%%%%%%%%%%%%%%%%%%%%%

\acknowledgments {\indent I have done my best in this dissertation to explain the research I have conducted as a linear progression from a set of mysteries and unanswered questions to questions to a set of conclusions. This style of explanation may be best for the sake of clarity, but it is not a faithful representation of the winding journey of discovery, both scientific and personal, that has characterized the past six years of my life. 

Throughout this journey, I have benefited from the guidance of many amazing scientists. I am truly lucky to have had the opportunity to work closely with Simon Birrer, who always seems to know the right question to ask in order to solve a problem. I am also indebted to Anna Nierenberg, with whom I have collaborated since day one. Anna has a keen an eye for important details others may overlook, and I believe working with her has made me a better scientist. 

It is difficult to convey just how much I admire my PhD advisor, Tommaso Treu. To the astronomy community as a whole, Tommaso is regarded as a productive, creative, and talented scientist. To me, he is a role model, a leader, and the kind of person I want to be. I could not have asked for a better advisor. 

Finally, I am thankful for my amazing parents. It must have been wild listening to me try to choose a career path: from a pilot, to joining the foreign service, and then finally setting down with astrophysics. Through it all, you have supported me, challenged me, and believed in me. As I get older, I realize more and more how lucky I am to have you two as my parents - I could not have done it without you! 
}

%%%%%%%%%%%%%%%%%%%%%%%%%%%%%%%%%%%%%%%%%%%%%%%%%%%%%%%%%%%%%%%%%%%%%%%%

\vitaitem   {2014}
                {B.S.~(Physics),
                James Madison University\\Phi Beta Kappa}
\vitaitem   {2016}
                {M.A.~(Physics), UCLA, Los Angeles, California.}

%%%%%%%%%%%%%%%%%%%%%%%%%%%%%%%%%%%%%%%%%%%%%%%%%%%%%%%%%%%%%%%%%%%%%%%%

\publication {{Gilman, D., et al. Constraints on the mass-concentration relation of cold dark matter halos with 11 strong gravitational lenses. MNRAS in press (2019)
		
\noindent Gilman, D., et al. Warm dark matter chills out: constraints on the halo mass function and the free-streaming length of dark matter with 8 quadruple-image strong gravitational lenses. MNRAS 491, 6077-6101 (2019)

\noindent Gilman, D., et al.  Probing dark matter structure down to $10^7$ solar masses: flux ratio statistics in gravitational lenses with line of sight halos. MNRAS 487, 5721-5738 (2019)

\noindent Gilman, D., et al. Probing the nature of dark matter by forward modelling flux ratios in strong gravitational lenses. MNRAS 481, 819-834 (2018)

\noindent Gilman, D., et al. Strong lensing signatures of luminous structure and substructure in early-type galaxies. MNRAS 467, 3970-3992 (2017) 

}}

%%%%%%%%%%%%%%%%%%%%%%%%%%%%%%%%%%%%%%%%%%%%%%%%%%%%%%%%%%%%%%%%%%%%%%%%

\abstract       {(Abstract omitted for brevity)}

%%%%%%%%%%%%%%%%%%%%%%%%%%%%%%%%%%%%%%%%%%%%%%%%%%%%%%%%%%%%%%%%%%%%%%%%



\begin {document}
\makeintropages

%%%%%%%%%%%%%%%%%%%%%%%%%%%%%%%%%%%%%%%%%%%%%%%%%%%%%%%%%%%%%%%%%%%%%%
%
% Ordinarily each chapter (at least) is in a separate file.
%
%\input {chapter1.tex}                         % Chapter 1 of dissertation
%\input {chapter2}                         % Chapter 2
%\input {chapter3}                         % etc.
%\input {chapter4}
%\input {chapter5}
%\input {chapter6}
%\input {chapter7}
%\input {chapter8}

\chapter{Introduction}

\chapter{Strong lensing signatures of luminous structure and substructure in early-type galaxies}
\textit{This chapter was published as Gilman, D., et al. Strong lensing signatures of luminous structure and substructure in early-type galaxies. MNRAS 467, 3970-3992 (2017), and is printed here with minor formatting adjustments.}

\chapter{Probing the nature of dark matter by forward modelling flux ratios in strong gravitational lenses}
\textit{This chapter was published as Gilman, D., et al. Probing the nature of dark matter by forward modelling flux ratios in strong gravitational lenses. MNRAS 481, 819-834 (2018), and is printed here with minor formatting adjustments.}

\chapter{Probing dark matter structure down to $10^7$ solar masses: flux ratio statistics in gravitational lenses with line of sight halos}
\textit{This chapter was published as Gilman, D., et al.  Probing dark matter structure down to $10^7$ solar masses: flux ratio statistics in gravitational lenses with line of sight halos. MNRAS 487, 5721-5738 (2019), and is printed here with minor formatting adjustments.}

\chapter{Warm dark matter chills out: constraints on the halo mass function and the free-streaming length of dark matter with 8 quadruple-image strong gravitational lenses}
\textit{This chapter was published as Gilman, D., et al. Warm dark matter chills out: constraints on the halo mass function and the free-streaming length of dark matter with 8 quadruple-image strong gravitational lenses. MNRAS 491, 6077-6101 (2019), and is printed here with minor formatting adjustments.}

\chapter{Constraints on the mass-concentration relation of cold dark matter halos with 11 strong gravitational lenses}
\textit{This chapter was published as Gilman, D., et al. Constraints on the mass-concentration relation of cold dark matter halos with 11 strong gravitational lenses. MNRAS in press (2019), and is printed here with minor formatting adjustments.}

%\bibliography {bib/network,bib/naming}    % bibliography references
%\bibliographystyle {thesis}

\end {document}

